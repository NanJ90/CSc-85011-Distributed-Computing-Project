\documentclass{ReportCUNY}
\usepackage{xspace}

\AssetPath{data/}
\CourseName{Distributed and Cloud Computing}
\CourseNumber{85011}
\CitationFile{./data/references.bib}
\DateD{31}
\DateM{10}
\DateY{2022}
\DepartmentAbbrev{CSc}
\DepartmentName{Computer Science}
\Institution{The Graduate Center~--~CUNY}
\Instructor{Saptarshi Debroy}
\Student{Isa Jafarov, Nan Jia, Alex Washburn, Rose Wong}
\Subtitle{Course Project Plan}
\Title{Project Name Goes Here}

\newcommand{\Nat}{\ensuremath{\mathbb{N}}\xspace}
\newcommand{\PosInt}{\ensuremath{\mathbb{Z}^{+}}\xspace}
\newcommand{\KeyWord}[1]{\textbf{\texttt{#1}}}
\newcommand{\KeyValuePair}[2]{\KeyWord{#1}\;:\qquad#2}
\newcommand{\KeyTypeValuePair}[3]{\KeyWord{#1}:~\makebox[2cm][l]{#2}~\qquad#3}

\begin{document}
\setlength{\belowdisplayskip}{0pt} \setlength{\belowdisplayshortskip}{1pt}
\setlength{\abovedisplayskip}{0pt} \setlength{\abovedisplayshortskip}{1pt}
\setlength{\abovedisplayskip}{5pt}
\setlength{\belowdisplayskip}{5pt}

\section{Introduction}

\section{Requirements}

\begin{itemize}
	\item Create a multi-cloud topology in GENI
	\begin{itemize}
		\item Multiple resource domains
		\item Different network and compute capacities
		\item Gateway users
		\item Submitting jobs with different performance requirements
		\item You can design dummy workloads or real AI/ML jobs
		\item Internal and external users
	\end{itemize}
	
	\item Implement a centralized brokering service for the Science Gateway
	\begin{itemize}
		\item Dynamic load balancer
		\item Replica management
		\item SDN for traffic engineering
		\item You can design from scratch or use container and/or K8s services
	\end{itemize}
	
\end{itemize}

\section{Design}

\begin{itemize}
	\item What are you trying to do? Articulate your objectives using absolutely no jargon.
	\item How is it done today, and what are the limits of current practice?
	\item What's new in your approach and why do you think it will be successful?
	\item Who cares? Think from both user side and service provider sides...
	\item If you're successful, what difference will it make?
	\item What major concepts related to the course were used and reinforced in the project?
\end{itemize}

\subsection{Jobs}

A \KeyWord{Job} contains the following information:

\begin{itemize}
\item \KeyTypeValuePair{Mail}{String}{An email address for the user, uniquely identifies user}
\item \KeyTypeValuePair{Time}{$\Nat$}{Hard upper bound limit for job, user provides best effort}
\item \KeyTypeValuePair{Disk}{$\text{MiB } \in \PosInt$}{Disk space requirements}
\item \KeyTypeValuePair{RAM~}{$\text{MiB } \in \PosInt$}{Memory requirements in MiB}
\item \KeyTypeValuePair{CPUs}{$\PosInt$}{number of CPU cores/threads}
\item \KeyTypeValuePair{GPUs}{$\Nat$}{GPU requirements}
\item \KeyTypeValuePair{Task}{Blob}{File of the executable to run}
\item \KeyTypeValuePair{Data}{Array Blob}{A list of data blobs to load into the disk space, total must be $\le \KeyWord{Disk}$}
\end{itemize}

The \KeyWord{Broker} can process a \KeyWord{Job}, and decide which resources to allocate to fulfill the job request across the \KeyWord{Domain}s.


\subsection{User Interface}

The \KeyWord{User} submits a \KeyWord{Job} at an endpoint.
The endpoint presents a User Interface (UI) to the \KeyWord{User}.
The presented UI could be a hosted website, a terminal user interface (TUI), or a standalone graphical user interface (GUI).
We will focus on a TUI for the initial implementation, with an HTML website UI as a stretch goal.

Required information of a \KeyWord{Job} is collected from the \KeyWord{User} by the endpoint UI.
Subsequently, the \KeyWord{Job} information is encoded as JSON by the endpoint and forwarded to the \KeyWord{Broker}.


\section{Work Delegation}

Who did what? Who did a lot of what?

\begin{itemize}
\item Isa Jafarov: \ldots
\item Nan Jia: \ldots
\item Alex Washburn: TUI endpoints
\item Rose Wong: \ldots
\end{itemize}


\section{Estimated Timeline}


\end{document}
