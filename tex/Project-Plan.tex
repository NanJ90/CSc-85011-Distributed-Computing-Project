\documentclass{ReportCUNY}
\usepackage{array}
\usepackage{xspace}

\AssetPath{data/}
\CourseName{Distributed and Cloud Computing}
\CourseNumber{85011}
\CitationFile{./data/references.bib}
\DateD{31}
\DateM{10}
\DateY{2022}
\DepartmentAbbrev{CSc}
\DepartmentName{Computer Science}
\Institution{The Graduate Center~--~CUNY}
\Instructor{Saptarshi Debroy}
\Student{Isa Jafarov, Nan Jia, Alex Washburn, Rose Wong}
\Subtitle{Course Project Plan}
\Title{Project Name Goes Here}

\newcommand{\Nat}{\ensuremath{\mathbb{N}}\xspace}
\newcommand{\PosInt}{\ensuremath{\mathbb{Z}^{+}}\xspace}
\newcommand{\KeyWord}[1]{\textbf{\texttt{#1}}}
\newcommand{\KeyValuePair}[2]{\KeyWord{#1}\;:\qquad#2}
\newcommand{\KeyTypeValuePair}[3]{\KeyWord{#1}:~\makebox[2cm][l]{#2}~\qquad#3}
\newcommand{\WorkloadStudentTaksPair}[2]{\makebox[2.75cm][l]{#1:}~#2}

\newcommand{\MonthDayFormat}[2]{\textbf{\texttt{#1}}~--~\textbf{\texttt{#2}}}

\begin{document}
\setlength{\belowdisplayskip}{0pt} \setlength{\belowdisplayshortskip}{1pt}
\setlength{\abovedisplayskip}{0pt} \setlength{\abovedisplayshortskip}{1pt}
\setlength{\abovedisplayskip}{5pt}
\setlength{\belowdisplayskip}{5pt}

\section{Introduction}

The course project comprises the creation of a distributed ``Science Broker'' which manages requests for scientific resources and information.
We will fulfill the project requirements by designing and implementation a service which receives requests for resources to satisfy a scientific job submission.
Furthermore, the service will be implemented in a distributed manner, permitting the sharing of resources across collaborating scientific ``domains.''


\section{Design}

\subsection{Overview: Utilize GENI multi-cloud topology}

We will use GENI to simulate multiple collaborating scientific institutions.
Each scientific organization will form a \KeyWord{Domain}.
A \KeyWord{Domain} is a network topology of resources contained entirely within a ``real'' GENI site.
Multiple \KeyWord{Domain} will be connected together within GENI to form a multi-could topology.
Every \ \KeyWord{Domain} has an \KeyWord{Endpoint} facilitating \KeyWord{User} access to the entire multi-could topology which comprises the Science Broker Service.
A \KeyWord{User} can submit a \KeyWord{Job} through an \KeyWord{Endpoint} by specifying the required resources.

Furthermore there will be an additional \KeyWord{Domain} containing containing the \KeyWord{Broker}.
All resources across all \KeyWord{Domain}s are known to the \KeyWord{Broker}.
Hence, the \KeyWord{Broker} handles scheduling, networking, and load balancing of submitted \KeyWord{Job}s.


\subsection{Broker}

\begin{center}\textbf{\texttt{T~O~D~O}}\end{center}

The \KeyWord{Broker} agent/instance is the centralized manager of the Science Broker Service.
Encapsulated within the \KeyWord{Broker} is the maintenance of an ACID database containing:
\begin{itemize}
\item A list of all incomplete \KeyWord{Job} submissions.
\item A list of all resources across each \KeyWord{Domain}.
\item A mapping of which resource(s) are allocated to which \KeyWord{Job}.
\item A queue of pending \KeyWord{Job} submissions.
\end{itemize}

Additionally, the \KeyWord{Broker} executes a scheduling algorithm to determine which queued \KeyWord{Job} will be given which resource(s) and when a \KeyWord{Job} will be migrated.


\subsection{Scheduling \& Queuing}

\begin{center}\textbf{\texttt{T~O~D~O}}\end{center}
Testing possible scheduling algorithms that are:
\item FCFS(first come first serve)
\item SJF(shortest job first)
\item RR(round robing)

Because we focus on monitoring our brokering network's resource management and performance. Those three algorithms have different advantages and disadvantages. 
\subsection{Domains}

\begin{center}\textbf{\texttt{T~O~D~O}}\end{center}


\subsection{Replication}

\begin{center}\textbf{\texttt{T~O~D~O}}\end{center}


\subsection{Jobs}

A \KeyWord{Job} contains the following information:

\begin{itemize}
\item \KeyTypeValuePair{Mail}{String}{An email address for the user, uniquely identifies user}
\item \KeyTypeValuePair{Time}{$\Nat$}{Hard upper bound limit for job, user provides best effort}
\item \KeyTypeValuePair{Disk}{$\text{MiB } \in \PosInt$}{Disk space requirements}
\item \KeyTypeValuePair{RAM~}{$\text{MiB } \in \PosInt$}{Memory requirements in MiB}
\item \KeyTypeValuePair{CPUs}{$\PosInt$}{number of CPU cores/threads}
\item \KeyTypeValuePair{GPUs}{$\Nat$}{GPU requirements}
\item \KeyTypeValuePair{Task}{Binary}{File of the executable to run}
\item \KeyTypeValuePair{Data}{Array Binary}{A list of data blobs to load into the disk space, total must be $\le \KeyWord{Disk}$}
\end{itemize}

The \KeyWord{Broker} can process a \KeyWord{Job}, and decide which resources to allocate to fulfill the job request across the \KeyWord{Domain}s.


\subsection{User Interface}

The \KeyWord{User} submits a \KeyWord{Job} at an \KeyWord{Endpoint}.
The \KeyWord{Endpoint} presents a User Interface (UI) to the \KeyWord{User}.
The presented UI could be a hosted website, a terminal user interface (TUI), or a standalone graphical user interface (GUI).
We will focus on a TUI for the initial implementation, with an HTML website UI as a stretch goal.

Required information of a \KeyWord{Job} is collected from the \KeyWord{User} by the \KeyWord{Endpoint} UI.
Subsequently, the \KeyWord{Job} information is encoded as JSON by the \KeyWord{Endpoint} and forwarded to the \KeyWord{Broker}.


\section{Work Delegation}

Main task ``umbrellas'' to be completed:

\begin{itemize}
\item Backend
\item Frontend
\item Containerization
\item Replication
\end{itemize}

Who will do what?

\begin{itemize}
\item \WorkloadStudentTaksPair{Isa Jafarov}{Backend Broker; centralized job/resource database, job queuing}
\item \WorkloadStudentTaksPair{Nan Jia}{Docker and Kubernetes startup scripts/containerization}
\item \WorkloadStudentTaksPair{Alex Washburn}{TUI \KeyWord{Endpoint}s, scheduling algorithm}
\item \WorkloadStudentTaksPair{Rose Wong}{Backend Backup/Replication}
\end{itemize}


\section{Estimated Timeline}

\begin{center}
\begin{tabular}{ |c| m{9.5cm}| l | c |} 
\hline
Week & Planned Work Items& Deliverable & Date\\
\hline
\MonthDayFormat{04}{16} & System design, GENI research & Project Plan & \MonthDayFormat{04}{23}\\ 
\MonthDayFormat{04}{23} & Frontend TUI, GENI experimentation, containerization & \makebox[6mm][l]{1st} Presentation & \MonthDayFormat{05}{01}\\
\MonthDayFormat{04}{30} & Replication, backups, scheduling algorithm & \makebox[2.85cm]{\hfill---~---\hfill} &  ---~--- \\
\MonthDayFormat{05}{07} & Backend UI, job status feeback, resource utilization  & \makebox[6mm][l]{2nd} Presentation &  \MonthDayFormat{05}{15} \\
\MonthDayFormat{05}{15} & Finalize everything! & Project Report &  \MonthDayFormat{05}{22} \\
\hline
\end{tabular}
\end{center}

\end{document}
